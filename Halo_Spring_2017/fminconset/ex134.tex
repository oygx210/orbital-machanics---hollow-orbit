\documentstyle[a4,11pt]{articl-n} 
\input{noindent} 
\begin{document} 
{\Large Eksempel 13.4 fra Optimization of Chemical Processes.}\\ 
  Redusert til 3 tanker. 
 
En har 3 ideelle blandetanker i serie.  I tankene foreg\aa r det 
en reaksjon gitt av $r = k c^{n}$ der $c$ er konsentrasjonen av den 
interessante komponenten.  Ved reaksjonen omdannes denne til andre 
stoffer.  Reaksjonen foreg\aa r ved konstant temperatur, og tettheten 
endres ikke som f\o lge av reaksjonen. 
 
Anlegget skal ha en gitt gjennomstr\o mning $q$ og det totale  
tank\-volumet skal v\ae re $V$.  Bestem volumene til de tre tankene 
slik at konsentrasjonen $c_{3}$ ut fra den siste tanken blir minst mulig. 
Kun stasjon\ae r\-verdier er av 
interesse. 
 
 
Oppgitt 
\begin{eqnarray} 
k & = & 0.00625 {\rm [ \frac{m^{3}}{kg \, mol} ]^{1.5} s^{-1}} \\ 
n & = & 2.5 \\ 
q & = & 0.01 {\rm \frac{m^{3}}{s}} \\ 
V & = & 10 {\rm m^{3}} \\ 
c_{0} & = & 20 {\rm \frac{kg \, mol}{m^{3}}} 
\end{eqnarray} 
 
En kan s\aa\ sette opp differensiallikningene for tankene ($i \in \{1\:2\:3\}$): 
\begin{eqnarray} 
\frac{dV_{i}}{dt} & = & q_{i-1} - q_{i} \\ 
\frac{d(V_{i}c_{i})}{dt} & = & q_{i-1}c_{i-1} - q_{i}c_{i} - V_{i}r_{i}  
\end{eqnarray} 
 
Vi er kun interessert i stasjon\ae re verdier og setter derfor de tidsderiverte 
lik 0.  Vi f\aa r da $q_{1}=q_{2}=q_{3}=q$.  Innf\o rer videre $\theta_{i} = 
V_{i}/q$ og $\Theta = V/q$.  Da f\aa r vi: 
\begin{eqnarray} 
0 & = & c_{0} - c_{1} - \theta_{1}r_{1} \\ 
0 & = & c_{1} - c_{2} - \theta_{2}r_{2} \\ 
0 & = & c_{2} - c_{3} - \theta_{3}r_{3} \\ 
0 & = & r_{1} - k c_{1}^{n} \\ 
0 & = & r_{2} - k c_{2}^{n} \\ 
0 & = & r_{3} - k c_{3}^{n} \\ 
0 & = & \theta_{1} + \theta_{2} + \theta_{3} - \Theta \\ 
\Theta & = & 1000 {\rm s} 
\end{eqnarray} 
Problemet kan derfor formuleres som 
\begin{equation} 
\min_{c_{1},c_{2},r_{1},r_{2},r_{3},\theta_{1},\theta_{2},\theta_{3}} c_{3} 
\end{equation} 
med (8) til (14) som bibetingelser. 
 
En kan la  
$c_{3}$ inng\aa\ i ${\bf x}$ 
og f\aa\ standardformen 
\begin{eqnarray} 
\min_{{\bf x}} f({\bf x}) \\ 
{\bf h (x)} & = & {\bf 0} 
\end{eqnarray} 
Der ${\bf x} = (c_{1},c_{2},c_{3},r_{1},r_{2},r_{3},\theta_{1},\theta_{2},\theta_{3})$ og 
$f({\bf x}) = x_{3} = c_{3}$.  H\o yre side av (8) til (14) utgj\o r 
${\bf h (x)}$. 
 
Hvordan skal dette l\o ses?  Hvordan veit vi at vi har funnet en 
optimal mulig l\o sning? 
\end{document} 


